\documentclass[10pt]{article}
\usepackage[brazil]{babel}
\usepackage{xcolor}
\usepackage{graphicx}
\usepackage{listingsutf8}
\usepackage{fontspec}
\usepackage{biblatex}
\usepackage[blocks]{authblk}
\usepackage[pdfborder={0 0 0}]{hyperref}
\addbibresource{main.bib}
\setmonofont{Fira Code}[Contextuals=Alternate]
\usepackage[verbatim]{lstfiracode}
\title{Sistemas Operacionais - TPIII}
\date{11 de Maio de 2023}
\author{
Gustavo Valadares Castro,
João Victor Martins Medeiros,
Hernane Velozo Rosa,
Pedro Igor Martins dos Reis}
\definecolor{keywordcolor}{RGB}{0,94,255}
\definecolor{stringcolor}{RGB}{200,50,100}
\definecolor{commentcolor}{RGB}{50,150,50}
\lstset{
    language=C,
    style=FiraCodeStyle,
    basicstyle=\ttfamily\small,
    keywordstyle=\color{keywordcolor}\bfseries,
    stringstyle=\color{stringcolor},
    commentstyle=\color{commentcolor},
    showstringspaces=false,
    breaklines=true,
    numbers=none,
    frame=single,
    rulecolor=\color{black},
    tabsize=4,
    captionpos=b,
    breakatwhitespace=false,
    extendedchars=true,
    inputencoding=utf8,
    morekeywords={printf, scanf, int, float, char, for, while, if, else},
}
\begin{document}
\maketitle
\section{Questão I}
\begin{lstlisting}[language=Java,caption=Programa 'Ping Pong']
package PingPong;
public class PingPong extends Thread
{
    private String palavra;
    public PingPong(String palavra){ this.palavra = palavra; }
    public void run()
    {
        try
        {
            for (int i = 0; i < 1000; i++)
            {
                System.out.print(palavra + "");
                if (palavra.equals("ping")) { Thread.sleep(1000); }
                else { Thread.sleep(500); }
            }
        }
        catch (InterruptedException e) { return; }
    public static void main(String[] args)
    {
        PingPong t1 = new PingPong("Ping ");
        PingPong t2 = new PingPong("Pong ");
        t1.start();
        t2.start();
    }
}
\end{lstlisting}
\begin{lstlisting}[language=Java,caption=Buffer.java]
package ProdutorConsumidor;
public class Buffer
{
    private int valor;
    private boolean cheio = false;
    public synchronized int get()
    {
        while (!cheio)
        {
            try { wait();}
            catch (InterruptedException e) {}
        }
        cheio = false;
        notifyAll();
        System.out.println("get " + valor);
        return valor;
    }
    public synchronized void put(int valor)
    {
        while (cheio)
        {
            try { wait(); }
            catch (InterruptedException e){}
        }
        this.valor = valor;
        cheio = true;
        System.out.println("put " + valor);
        notifyAll();
    }
}
\end{lstlisting}
\begin{lstlisting}[language=Java,caption=Consumidor.java]
public class Consumidor extends Thread
{
    private Buffer buf;
    private int numero;
    public Consumidor(Buffer buf, int numero)
    {
        this.buf = buf;
        this.numero = numero;
    }
    public void run()
    {
        int x;
        for (int i = 0; i < 5; i++)
        {
            x = buf.get();
            System.out.println("Consumidor #" + numero + "get: " + x);
        }
    }
}
\end{lstlisting}
\begin{lstlisting}[language=Java,caption=ProdCons.java]
public class ProdCons
{
    public static void main (String[] args)
    {
        Buffer buf = new Buffer();
        int i;
        for(i = 0; i < 1; i++)
        {
            new Consumidor(buf, i).start();
            new Produtor(buf, i).start();
        }
    }
}
\end{lstlisting}
\begin{lstlisting}[language=Java,caption=Produtor.java]
public class Produtor extends Thread
{
        private Buffer buf;
        private int numero;
        public Produtor(Buffer buf, int numero)
        {
            this.buf = buf;
            this.numero = numero;
        }
        public void run()
        {
            for(int i = 0; i < 5; i++)
            {
                buf.put(i);
                System.out.println("Produtor #" + numero + " put: " + i);
                try { sleep((int) (Math.random() * 100)); } catch (InterruptedException e) {}
            }
        }
    }
\end{lstlisting}
Este algoritmo composto pelas classes \verb|Consumidor|, \verb|ProdCons| e \verb|Produtor| basicamente simula uma compra/venda onde "Consumidor" e "Produtor" dividem um buffer de tamanho fixo onde o Produtor insere informações ou enfilera jobs no buffer e o consumidor as retira do buffer ou "compra os produtos", como ambas as threads estão sendo executadas simultaneamente elas precisam saber quando houve alteração no buffer assim entra o nosso \verb|notifyAll| tendo como precedente o wait fazendo com que as threads aguardem até serem notificadas de alteração.
\section{Questão II}
\subsection{Problema do produtor e consumidor}
\begin{lstlisting}[caption=Código fornecido pelo enunciado]
#include <sys/time.h>
#include <sys/types.h>
#include <pthread.h>
#include <semaphore.h>
#include <stdlib.h>
#include <stdio.h>
#define MAX 20 // tamanho da fila circular
#define ROUND 100 // numero de iteracoes
#define DRES 500000
char buffer[MAX]; // fila circular
int inptr=0, outptr=0, inbuffer=0; // apontadores para a fila
struct timespec delay;
struct timeval tv;
pthread_mutex_t mutex; // mutex
sem_t s_produtor, s_consumidor; // semaforos para produtor e consumidor
// Exibe o conteudo da fila e o valor dos semaforos
void mostrabuffer(char *mens)
{
    int i, s_pro, s_con;
    printf("%-10s ", mens);
    for (i=0; i<MAX; i++)
    {
        printf("%c", buffer[i]);
    }
    sem_getvalue(&s_produtor, &s_pro);
    sem_getvalue(&s_consumidor, &s_con);
    printf(" p=%03d c=%03d (%03d)\n", s_pro, s_con, inbuffer);
}
// insere um novo caractere na fila
void push(char c)
{
    if (inbuffer<MAX)
    {
        pthread_mutex_lock(&mutex); // inicio da sessao critica
        buffer[inptr]=c;
        inptr=(inptr+1)%MAX;
        inbuffer++;
        pthread_mutex_unlock(&mutex); // final da sessao critica
    }
}
// retira um caractere da fila
char pop(void)
{
    char c;
    if (inbuffer>0) {
        pthread_mutex_lock(&mutex); // inicio da sessao critica
        c=buffer[outptr];
        buffer[outptr]=' ';
        outptr=(outptr+1)%MAX;
        inbuffer--;
        pthread_mutex_unlock(&mutex); // final da sessao critica
    }
    return(c);
}
// Produtor: sorteia uma letra e insere na fila
void produtor(void *ptr)
{
    int n=ROUND;
    char c;
    while (n>0)
    {
        sem_wait(&s_produtor); // aguarda um espaco na fila
        c=rand()%26+'A';
        push(c);
        mostrabuffer("push");
        n--;
        sem_post(&s_consumidor);
        tv.tv_sec=0;
        tv.tv_usec=((int)rand())%DRES;
        select(0, NULL, NULL, NULL, &tv); // delay randomico
    }
    pthread_exit(0);
}
// Consumidor: retira da fila um elemento e mostra na tela
void consumidor(void *ptr)
{
    int n=ROUND;
    char c;
    while (n>0)
    {
        sem_wait(&s_consumidor); // aguarda um item na fila
        c=pop();
        mostrabuffer("pop");
        n--;
        sem_post(&s_produtor);
        tv.tv_sec=0;
        tv.tv_usec=((int)rand())%DRES;
        select(0, NULL, NULL, NULL, &tv); // delay randomico
    }
    pthread_exit(0);
}
/* Programa principal - cria duas threads, uma produtora e
* outra consumidora e aguarda a finalizacao de ambas */
int main()
{
    pthread_t produtor_t, consumidor_t;
    char *message1 = "Hello";
    char *message2 = "World";
    int i;
    srand(time(NULL));
    for (i=0; i<MAX; i++)
    buffer[i]=' '; // prepara a fila
    sem_init(&s_produtor, 0, MAX); // produtor inicia com semaforo no
    maximo
    sem_init(&s_consumidor, 0, 0); // consumidor inicia com semaforo = 0
    mostrabuffer("init");
    // cria threads
    pthread_create(&produtor_t, NULL, (void*)&produtor, NULL);
    pthread_create(&consumidor_t, NULL, (void*)&consumidor, NULL);
    // aguarda o final das threads
    pthread_join(produtor_t, NULL);
    pthread_join(consumidor_t, NULL);
    mostrabuffer("end");
    sem_destroy(&s_produtor);
    sem_destroy(&s_consumidor);
    return 0;
}
\end{lstlisting}
A ideia do código é que existe um produtor e um consumidor que compartilham um buffer comum. O produtor produz algo e o coloca no buffer, enquanto o consumidor consome o item do buffer. Para evitar condições de corrida, é feito um mutex (exclusão mútua) e semáforos.
Sobre o código em si:
\begin{itemize}
\item \textbf{Variáveis Globais e Definições:} Aqui, declara-se o buffer (um array de caracteres) que será compartilhado entre o produtor e o consumidor. Também são definidos um mutex e dois semáforos, um para o produtor \verb|(s_produtor)| e outro para o consumidor \verb|(s_consumidor)|. Inicializam-se os ponteiros para o início e o fim do buffer e a quantidade de itens no buffer (inbuffer).
\item \textbf{Funções push e pop:} A função push insere um item no buffer e a função pop remove um item do buffer. Essas funções são protegidas por um mutex para evitar condições de corrida.
\item \textbf{Funções produtor e consumidor:} O produtor gera um caractere aleatório e o coloca no buffer usando a função push. Antes de colocar um item no buffer, o produtor espera \verb|sem_wait| para garantir que haja espaço no buffer.
Depois de colocar um item no buffer, ele sinaliza \verb|sem_post| para o consumidor que há um item disponível.
O consumidor faz o oposto: espera um item estar disponível no buffer e depois o consome. Ele então sinaliza para o produtor que há espaço disponível no buffer.
\item \textbf{Função main:} Nesta função, inicializam-se os semáforos e o buffer, e são criadas as threads para o produtor e o consumidor. A thread do produtor é inicializada para o tamanho máximo do buffer e a thread do consumidor é inicializada para 0, pois inicialmente o buffer está vazio. Em seguida, é esperado que as threads terminem com a função \verb|pthread_join| e, finalmente, os semáforos são destruídos.
\end{itemize}
O código também inclui a função \verb|mostrabuffer| para exibir o estado atual do buffer e dos semáforos, o que é útil para depuração e entendimento do programa. Além disso, ele introduz um atraso aleatório após cada operação de produção e consumo para tornar a interação entre o produtor e o consumidor mais realista.
\subsection{Problema do Barbeiro Dorminhoco}
O código abaixo cria uma thread para o barbeiro e várias threads para os clientes.
O barbeiro e os clientes operam em loops infinitos, onde o barbeiro continua cortando o cabelo
dos clientes e os clientes continuam chegando para ter seus cabelos cortados. O semáforo "clientes"
é usado para representar o número de clientes esperando (inicialmente 0). O semáforo "barbeiros"
é usado para representar o número de barbeiros disponíveis (inicialmente 0). O semáforo "mutex" é
usado para garantir a exclusão mútua quando o número de clientes esperando ("waiting") é alterado.
\begin{lstlisting}
#include <pthread.h>
#include <semaphore.h>
#include <stdio.h>
#include <stdlib.h>
#include <unistd.h>
#define CHAIRS 5    // Número de cadeiras para os clientes esperarem.
sem_t customers;    // Número de cliente esperando por serviço.
sem_t barbers;      // Número de barbeiros esperando por clientes.
sem_t mutex;        // Para exclusão mútua.
int waiting = 0;    // Clientes que estão esperando (não estão sendo cortados).
// Protótipos.
void* barber(void *arg);
void* customer(void *arg);
int main()
{
    pthread_t b, c;
    // Inicializa os semáforos.
    sem_init(&customers, 0, 0);
    sem_init(&barbers, 0, 0);
    sem_init(&mutex, 0, 1);
    // Cria o barbeiro.
    pthread_create(&b, NULL, barber, NULL);
    // Cria os clientes.
    for(int i = 0; i < CHAIRS+1; i++)
    {
        pthread_create(&c, NULL, customer, NULL);
        sleep(1);
    }
    // Junta as threads.
    pthread_join(b, NULL);
    pthread_join(c, NULL);
    return 0;
}
void* barber(void *arg)
{
    while(1)
    {
        sem_wait(&customers);   // Tenta adquirir um cliente.
        sem_wait(&mutex);       // Tenta adquirir acesso a 'waiting'.
        waiting = waiting - 1;  // Decrementa a contagem de clientes esperando.
        sem_post(&barbers);     // Um barbeiro agora está pronto para cortar o cabelo.
        sem_post(&mutex);       // Libera 'waiting'.
        printf("O barbeiro está cortando o cabelo de um cliente\n");
        sleep(3);               // Gasta um tempo cortando o cabelo.
    }
}
void* customer(void *arg)
{
    sem_wait(&mutex);           // Tenta adquirir acesso a 'waiting'.
    if(waiting < CHAIRS)
    {      // Se não houver cadeiras vazias, saia.
        printf("Um cliente chegou e está esperando\n");
        waiting = waiting + 1;  // Incrementa a contagem de clientes à espera.
        sem_post(&customers);   // Acorda o barbeiro se necessário.
        sem_post(&mutex);       // Libera o acesso a 'waiting'.
        sem_wait(&barbers);     // Vai dormir se o número de barbeiros livres for 0.
        printf("O cliente está tendo seu cabelo cortado\n");
    }
    else
    {
        // A barbearia está cheia; não espera.
        sem_post(&mutex);
        printf("A barbearia está cheia; o cliente está saindo.\n");
    }
}
\end{lstlisting}
As principais funções do programa são \verb|'barber'| e \verb|'customer'|, sendo que:
\verb|barber(void *arg)| representa o barbeiro. O barbeiro espera por um cliente \verb|(sem_wait(&customers))|.
Quando um cliente chega, o barbeiro começa a cortar o cabelo do cliente. O acesso à variável "waiting" é protegido
por um mutex para evitar condições de corrida.
Já a função \verb|customer(void *arg)| representa um cliente. O cliente primeiro adquire o mutex para verificar se há
uma cadeira vazia. Se todas as cadeiras estiverem ocupadas (ou seja, waiting == CHAIRS), o cliente deixa a barbearia. Se
houver uma cadeira vazia, o cliente se senta na cadeira e libera o mutex. Em seguida, o cliente acorda o barbeiro (se necessário)
usando \verb|sem_post(&customers)|. Por fim, o cliente espera pelo barbeiro para cortar seu cabelo \verb|(sem_wait(&barbers))|.
\subsection{Problema dos Leitores e Escritores}
O problema descreve um sistema onde um conjunto de dados é compartilhado
entre várias threads. Algumas dessas threads podem apenas ler os dados
(leitores) enquanto outras threads podem tanto ler quanto alterar os dados (escritores).
\begin{lstlisting}
#include <pthread.h>
#include <semaphore.h>
#include <stdio.h>

#define TAM_ARRAY 5
#define ITERACOES 5
int array[TAM_ARRAY];

sem_t mutex, db;
int rc = 0;

void *
leitor (void *arg)
{
  int i, j;
  for (j = 0; j < ITERACOES; j++)
    {
      sem_wait (&mutex);
      rc++;
      if (rc == 1)
    sem_wait (&db);
      sem_post (&mutex);
      printf ("\t → Leitor %ld lendo: ", (long) arg);
      for (i = 0; i < TAM_ARRAY; i++)
    {
      printf ("%d ", array[i]);
    }
      printf ("\n");
      sem_wait (&mutex);
      rc--;
      if (rc == 0)
    sem_post (&db);
      sem_post (&mutex);
    }
}

void *
escritor (void *arg)
{
  int i, j;
  for (j = 0; j < ITERACOES; j++)
    {
      sem_wait (&db);
      printf ("\t → Escritor %ld escrevendo\n", (long) arg);
      for (i = 0; i < TAM_ARRAY; i++)
    {
      array[i] = (long) arg;
    }
      sem_post (&db);
    }
}

int
main ()
{
  int i;
  int n_leitores = 0;
  int n_escritores = 0;
  int n_interacoes = 0;
  printf("\n\t Defina a quantidade de leitores: ");
  scanf("%d", &n_leitores);
  printf("\n\t Defina a quantidade de escritores: ");
  scanf("%d", &n_escritores);

  pthread_t leitores[n_leitores];
  pthread_t escritores[n_escritores];

  sem_init (&mutex, 0, 1);
  sem_init (&db, 0, 1);

  for (i = 0; i < n_leitores; i++)
    {
      pthread_create (&leitores[i], NULL, leitor, (void *) i);
    }
  for (i = 0; i < n_escritores; i++)
    {
      pthread_create (&escritores[i], NULL, escritor, (void *) i);
    }
  for (i = 0; i < n_leitores; i++)
    {
      pthread_join (leitores[i], NULL);
    }
  for (i = 0; i < n_escritores; i++)
    {
      pthread_join (escritores[i], NULL);
    }

  return 0;
}
\end{lstlisting}
Neste código, há um semáforo mutex usado para acesso exclusivo ao contador \verb|rc| (número de leitores lendo atualmente), e um semáforo \verb|db| usado para acesso exclusivo
ao banco de dados.
A função \verb|leitor| adquire o semáforo mutex, incrementa o contador de leitores \verb|rc| e, se for o primeiro leitor, adquire o semáforo \verb|db|. Depois de ler, o leitor decrementa o contador \verb|rc| e, se for o último leitor, libera o semáforo \verb|db|.
A função \verb|escritor| apenas adquire o semáforo \verb|db|, escreve e depois o libera.
\subsection{Problema dos Fumantes}
A situação é de três fumantes sentados em uma mesa, cada um com um dos três ingredientes: tabaco, papel e fósforos. Há também um agente que coloca dois dos três ingredientes na mesa. O fumante que tem o terceiro ingrediente faltante então faz e fuma um cigarro, sinalizando o agente ao terminar. O agente então coloca outro dois dos três ingredientes na mesa e o ciclo continua.
\begin{lstlisting}
#include <pthread.h>
#include <semaphore.h>
#include <stdlib.h>
#include <stdio.h>
#include <unistd.h>
sem_t sem_agent;
sem_t sem_smoker[3];
void *agent(void *arg);
void *smoker(void *arg);
int main()
{
    pthread_t agent_thread, smokers[3];
    sem_init(&sem_agent, 0, 1);
    for(int i = 0; i < 3; i++) { sem_init(&sem_smoker[i], 0, 0); }
    pthread_create(&agent_thread, NULL, agent, NULL);
    for(int i = 0; i < 3; i++) { pthread_create(&smokers[i], NULL, smoker, (void *)i); }
    pthread_join(agent_thread, NULL);
    for(int i = 0; i < 3; i++) { pthread_join(smokers[i], NULL); }
    return 0;
}
void *agent(void *arg)
{
    int ingredient;
    while(1)
    {
        sem_wait(&sem_agent);
        ingredient = rand() % 3;
        printf("Agente colocou ingredientes na mesa.\n");
        sem_post(&sem_smoker[ingredient]);
        sleep(1);
    }
}
void *smoker(void *arg)
{
    int smoker_id = (int)arg;
    while(1)
    {
        sem_wait(&sem_smoker[smoker_id]);
        printf("Fumante %d fez e fumou o cigarro.\n", smoker_id);
        sem_post(&sem_agent);
        sleep(1);
    }
}
\end{lstlisting}
Neste código, há um semáforo \verb|sem_agent| que o agente usa para sincronizar com os fumantes. Os semáforos \verb|sem_smoker[i]| são usados pelos fumantes para sincronizar com o agente.
A função \verb|agent| aguarda o semáforo \verb|sem_agent|, coloca ingredientes na mesa e sinaliza o fumante que tem o ingrediente faltante.
A função \verb|smoker| aguarda o semáforo \verb|sem_smoker[i]|, faz e fuma um cigarro e sinaliza o agente.
\subsection{Problema dos Filósofos}
Neste problema, cinco filósofos silenciosos sentam à mesa em torno de um prato de espaguete. Um garfo está colocado entre cada par de filósofos adjacentes (totalizando 5 garfos). Cada filósofo pode alternar entre duas operações: comer ou pensar. No entanto, cada filósofo precisa de ambos os garfos para pegar o espaguete e comer. Se um filósofo não puder pegar ambos os garfos (porque pelo menos um está com o vizinho), ele ficará esperando até poder pegar os dois.

Em um cenário onde todos os filósofos pegarem o garfo ao mesmo tempo, pode gerar o deadlock, uma forma de contornar este problema seria adicionar uma garçom (sendo outro semáforo), através deste, serão limitados \textit{N - 1} filósofos recolherem os garços, conforme código abaixo:
\begin{lstlisting}
#include <pthread.h>
#include <semaphore.h>
#include <stdio.h>
#define N 5
#define PENSANDO 0
#define FAMINTO 1
#define COMENDO 2
#define ESQUERDA (numero_filosofo + 4) % N
#define DIREITA (numero_filosofo + 1) % N
sem_t mutex;
sem_t S[N];
void *filosofo(void *num);
void pegar_garfos(int);
void devolver_garfos(int);
void testar(int);
int estado[N];
int filosofo_num[N]={0,1,2,3,4};
int main()
{
    int i;
    pthread_t thread_id[N];
    sem_init(&mutex,0,1);
    for(i=0; i<N; i++) { sem_init(&S[i],0,0); }
    for(i=0; i<N; i++)
    {
        pthread_create(&thread_id[i], NULL, filosofo, &filosofo_num[i]);
        printf("Filosofo %d esta pensando\n",i+1);
    }
    for(i=0; i<N; i++) { pthread_join(thread_id[i],NULL); }
}
void *filosofo(void *num)
{
    while(1)
    {
        int *i = num;
        sleep(1);
        pegar_garfos(*i);
        sleep(0);
        devolver_garfos(*i);
    }
}
void pegar_garfos(int numero_filosofo)
{
    sem_wait(&mutex);
    estado[numero_filosofo] = FAMINTO;
    printf("Filosofo %d esta faminto\n", numero_filosofo + 1);
    testar(numero_filosofo);
    sem_post(&mutex);
    sem_wait(&S[numero_filosofo]);
    sleep(1);
}
void devolver_garfos(int numero_filosofo)
{
    sem_wait(&mutex);
    estado[numero_filosofo]= PENSANDO;
    printf("Filosofo %d devolveu os garfos %d e %d\n", numero_filosofo + 1, ESQUERDA + 1, numero_filosofo + 1);
    printf("Filosofo %d esta pensando\n", numero_filosofo + 1);
    testar(ESQUERDA);
    testar(DIREITA);
    sem_post(&mutex);
}
void testar(int numero_filosofo)
{
    if(estado[numero_filosofo] == FAMINTO && estado[ESQUERDA] != COMENDO && estado[DIREITA] != COMENDO)
    {
        estado[numero_filosofo] = COMENDO;
        sleep(2);
        printf("Filosofo %d pegou os garfos %d %d e %d\n", numero_filosofo + 1, ESQUERDA + 1, numero_filosofo + 1);
        printf("Filosofo %d esta comendo\n", numero_filosofo + 1);
        sem_post(&S[numero_filosofo]);
    }
}
\end{lstlisting}
No algoritmo acima, cada filósofo é representado por uma thread. Cada garfo é representado como um semáforo.
\begin{itemize}
\item A função \verb|filosofo| é a rotina de cada thread. Em um loop infinito, um filósofo pensa, pega os garfos, come e depois coloca os garfos de volta;
\item A função \verb|pegar_garfos| é chamada quando um filósofo quer comer. Primeiro, ele informa que está faminto (muda seu estado para \verb|FAMINTO|), tenta pegar os garfos e, se não conseguir, espera até que possa;
\item A função \verb|devolver_garfos| é chamada quando um filósofo termina de comer. Ele muda seu estado para \verb|PENSANDO|, coloca os garfos de volta (libera os semáforos) e, em seguida, permite que seus vizinhos (os filósofos à sua esquerda e à sua direita) tentem pegar os garfos;
\item A função \verb|testar| é chamada para verificar se um filósofo pode começar a comer. Um filósofo pode comer se estiver faminto e se seus vizinhos não estiverem comendo.
\end{itemize}
\nocite{*}
\printbibliography
\end{document}

\documentclass[12pt]{article}
\usepackage[brazil]{babel}
\usepackage{xcolor}
\usepackage{graphicx}
\usepackage{listingsutf8}
\usepackage{fontspec}
\setmonofont{Fira Code}[Contextuals=Alternate]
\usepackage{listings}
\usepackage[verbatim]{lstfiracode}
\title{Sistemas Operacionais - Atv. Extra}
\author{Pedro Igor Martins dos Reis}
\date{12 de Maio de 2023}
\definecolor{gruvbox_bg}{HTML}{282828}
\definecolor{gruvbox_fg}{HTML}{EBDBB2}
\definecolor{gruvbox_comment}{HTML}{928374}
\definecolor{gruvbox_string}{HTML}{B8BB26}
\definecolor{gruvbox_keyword}{HTML}{458588}
\definecolor{gruvbox_function}{HTML}{83A598}
\definecolor{gruvbox_variable}{HTML}{D79921}
\lstset{
  language=C,
  style=FiraCodeStyle,
  backgroundcolor=\color{gruvbox_bg},
  frame=single,
  captionpos=b,
  numbers=left,
  breaklines=true,
  morekeywords={int, char, float, double, if, else, while, for, return},
  morekeywords=[2]{printf, scanf, main},
  morekeywords=[3]{\#include, \#define},
  basicstyle=\color{gruvbox_fg}\ttfamily\tiny,
  commentstyle=\color{gruvbox_comment},
  stringstyle=\color{gruvbox_string},
  keywordstyle=\color{gruvbox_keyword},
  keywordstyle=[2]\color{gruvbox_function},
  keywordstyle=[3]\color{gruvbox_variable},
  sensitive=true,
  numberstyle=\tiny\color{gruvbox_comment},
  morekeywords={int, char, float, double, if, else, while, for, return},
  morekeywords=[2]{printf, scanf, main},
  morekeywords=[3]{\#include, \#define},
  rulecolor=\color{gruvbox_bg},
  showstringspaces=false,
}
\lstset{
  literate=%
    {0}{{{\color{gruvbox_variable}0}}}1
    {1}{{{\color{gruvbox_variable}1}}}1
    {2}{{{\color{gruvbox_variable}2}}}1
    {3}{{{\color{gruvbox_variable}3}}}1
    {4}{{{\color{gruvbox_variable}4}}}1
    {5}{{{\color{gruvbox_variable}5}}}1
    {6}{{{\color{gruvbox_variable}6}}}1
    {7}{{{\color{gruvbox_variable}7}}}1
    {8}{{{\color{gruvbox_variable}8}}}1
    {9}{{{\color{gruvbox_variable}9}}}1,
  morestring=[b]",
  stringstyle=\color{gruvbox_variable},
  literate=%
    {*}{{{\color{gruvbox_variable}*}}}1
    {/}{{{\color{gruvbox_variable}/}}}1
    {\&}{{{\color{gruvbox_variable}\&}}}1
    {=}{{{\color{gruvbox_variable}=}}}1
    {-}{{{\color{gruvbox_variable}-}}}1
    {+}{{{\color{gruvbox_variable}+}}}1
    {!}{{{\color{gruvbox_variable}!}}}1
    {(}{{{\color{gruvbox_variable}(}}}1
    {)}{{{\color{gruvbox_variable})}}}1
    {[}{{{\color{gruvbox_variable}[}}}1
    {]}{{{\color{gruvbox_variable}]}}}1
    {<}{{{\color{gruvbox_variable}<}}}1
    {>}{{{\color{gruvbox_variable}>}}}1
    {.}{{{\color{gruvbox_variable}.}}}1
    {;}{{{\color{gruvbox_variable};}}}1
    {,}{{{\color{gruvbox_variable},}}}1
    {?}{{{\color{gruvbox_variable}?}}}1,
}
\begin{document}
\maketitle
\section{Questão III}
No problema clássico \textbf{'O jantar dos Selvagens'}, em que cada
\textit{selvagem} e o \textit{cozinheiro} serão threads e os semáforos serão usados para
controlar o acesso ao caldeirão e a sincronização entre os selvagens e o cozinheiro.
\begin{lstlisting}[caption=O jantar dos Selvagens,style=FiraCodeStyle]
#include <pthread.h>
#include <semaphore.h>
#include <stdio.h>
#define N 5                 // Número de porções no caldeirão.
sem_t sem_selvagem;         // Semaforo para controlar a vez dos selvagens.
sem_t sem_cozinheiro;       // Semaforo para controlar a vez do cozinheiro.
int porcoes = N;            // Número inicial de porções no caldeirão.
void* selvagem(void* arg)   // Função executada pela thread do selvagem.
{
    while (1)
    {
        sem_wait(&sem_selvagem); // Espera sua vez de se servir.
        porcoes--; // Consome uma porção.
        printf("Selvagem comeu uma porção. Restam %d porções.\n", porcoes);
        if (porcoes == 0)
        {
            printf("Caldeirão vazio. Selvagem acorda o cozinheiro.\n");
            sem_post(&sem_cozinheiro); // Acorda o cozinheiro se o caldeirão estiver vazio.
        }
        else { sem_post(&sem_selvagem); } // Passa a vez para o próximo selvagem se ainda houver porções.
    }
}
void* cozinheiro(void* arg) // Função executada pela thread do cozinheiro.
{
    while (1)
    {
        sem_wait(&sem_cozinheiro); // Espera ser acordado.
        porcoes = N; // Enche o caldeirão.
        printf("Cozinheiro encheu o caldeirão. Há %d porções.\n", porcoes);
        sem_post(&sem_selvagem); // Acorda o próximo selvagem.
    }
}
int main()
{
    pthread_t coz_thread, selv_thread;
    sem_init(&sem_selvagem, 0, 1); // Inicializa o semáforo do selvagem com 1 para que ele possa comer primeiro.
    sem_init(&sem_cozinheiro, 0, 0); // Inicializa o semáforo do cozinheiro com 0 pois ele só deve cozinhar quando o caldeirão estiver vazio.
    pthread_create(&selv_thread, NULL, selvagem, NULL); // Cria a thread do selvagem.
    pthread_create(&coz_thread, NULL, cozinheiro, NULL); // Cria a thread do cozinheiro.
    pthread_join(selv_thread, NULL); // Espera a thread do selvagem terminar.
    pthread_join(coz_thread, NULL); // Espera a thread do cozinheiro terminar.
    return 0;
}
\end{lstlisting}
No código acima, temos um semáforo para o cozinheiro \verb|(sem_cozinheiro)|
e um para os selvagens \verb|(sem_selvagem)|. Quando um selvagem pega a
última porção, ele acorda o cozinheiro para reabastecer o caldeirão.
O cozinheiro então enche o caldeirão e volta a dormir, liberando o semáforo para os selvagens se servirem novamente.
Desta forma, \textbf{o problema dos três robôs} pode ser resolvido utilizando
semáforos para manter a ordem de execução das threads de cada robô.
Em suma, a ideia básica é que cada robô (thread) vai esperar seu semáforo
antes de se mover, e depois vai liberar o semáforo do próximo robô na sequência.
\begin{lstlisting}[caption=Problema dos três robôs]
#include <pthread.h>
#include <semaphore.h>
#include <stdio.h>
sem_t sem_bart, sem_lisa, sem_maggie; // Semaforos para cada robô.
void move(char *name) // Função que cada robô executará.
{
    printf("%s está se movendo...\n", name);
}
void* bart(void* arg) { // Função executada pela thread de Bart.
    while (1) {
        sem_wait(&sem_bart); // Bart espera sua vez.
        move("Bart"); // Bart se move.
        sem_post(&sem_lisa); // Bart acorda Lisa.
    }
}
void* lisa(void* arg) // Função executada pela thread de Lisa.
{
    while (1)
    {
        sem_wait(&sem_lisa); // Lisa espera sua vez.
        move("Lisa"); // Lisa se move.
        sem_post(&sem_maggie); // Lisa acorda Maggie.
    }
}
void* maggie(void* arg) // Função executada pela thread de Maggie.
{
    while (1)
    {
        sem_wait(&sem_maggie); // Maggie espera sua vez.
        move("Maggie"); // Maggie se move.
        sem_post(&sem_lisa); // Maggie acorda Lisa.
    }
}
int main()
{
    pthread_t bart_thread, lisa_thread, maggie_thread;
    // Inicializa os semáforos. Bart começa se movendo, então seu semáforo é inicializado com 1.
    // Lisa e Maggie esperam Bart, então seus semáforos são inicializados com 0.
    sem_init(&sem_bart, 0, 1);
    sem_init(&sem_lisa, 0, 0);
    sem_init(&sem_maggie, 0, 0);
    // Cria as threads para cada robô.
    pthread_create(&bart_thread, NULL, bart, NULL);
    pthread_create(&lisa_thread, NULL, lisa, NULL);
    pthread_create(&maggie_thread, NULL, maggie, NULL);
    // Espera as threads terminarem.
    pthread_join(bart_thread, NULL);
    pthread_join(lisa_thread, NULL);
    pthread_join(maggie_thread, NULL);
    return 0;
}
\end{lstlisting}
Cada função de thread (\verb|bart|, \verb|lisa|, \verb|maggie|)
representa o loop de execução de cada robô. Cada robô
espera por seu semáforo, se move, e então libera o semáforo
do próximo robô na sequência. Como \textit{Lisa} se move duas vezes na sequência, o mesmo terá duas chamadas para \verb|sem_wait| e \verb|sem_post|.
\end{document}
